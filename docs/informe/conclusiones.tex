\part{Conclusiones}
\label{parte_conclusiones}
El resultado de este trabajo es un simulador de la arquitectura Q que corre en un navegador y que puede ser usado por los y las estudiantes en 
el contexto de la materia Organización de Computadoras; permitiéndoles probar los programas que escriben, sin tener que simularlos en papel.
Además, fomentó la curiosidad, ya que realizaron programas por fuera de las prácticas y pusieron a prueba sus conocimientos. 

Desde el desarrollo de la herramienta notamos un aprendizaje en términos didácticos ya que habitualmente el software que realizamos en 
industria no es apuntado a la enseñanza. Sabemos que queda mucho por mejorar en este aspecto pero quienes hicimos este trabajo tenemos 
la intención de continuarlo para poder dar mejores funcionalidades tanto a estudiantes como a docentes.

Uno de los desafíos era la construcción de la idea de lenguaje, mediante el uso de gramáticas y parseo, aprendimos lo necesario para poder
completar el trabajo y seguramente sigamos aprendiendo a lo largo de la licenciatura.

Construimos una librería pensando en evitar siempre el acoplamiento a nuestra interfaz de usuario e incluso a nuestro parser, creemos que 
puede ser realmente el puntapie inicial para que otros/as estudiantes de la carrera puedan intervenir y mejorar el trabajo realizado.

\section{Trabajos futuros}
Encontramos durante el desarrollo algunas ideas que pueden incluirse a futuro:
\begin{itemize}
  \item Manejo de instrucciones I/O. Actualmente la arquitectura Q no define estas instrucciones pero podría ser una extensión que aporte
  interactividad a los programas.
  \item Permitir arquitecturas de computadoras distintas a Q, por ejemplo, poder variar el tamaño del bus de datos o direcciones.
  \item Incluir algún método de revisión docente, como por ejemplo deep link o envío de mails.
  \item Instrucción NOOP, para poder terminar programas sin necesariamente realizar una acción como última instrucción.
  \item Permitir la inicialización de registros y memoria antes de la ejecución.
  \item Mejorar el manejo de errores para mostrar mensajes más descriptivos.
\end{itemize}
