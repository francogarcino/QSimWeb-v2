\subsubsection{ADD}
Representa la instrucción ADD de la arquitectura Q. Es subclase de FlagInstruction. Sobreescribe los siguientes métodos:
\begin{itemize}
\item perform\_operation: recibe los operandos de origen y destino y devuelve la suma de los mismos.
\item calculate\_overflow: recibe los operandos de origen y destino y el resultado de la operación. Devuelve verdadero si se cumple alguna de las siguientes condiciones: 
	\begin{itemize}
	\item Los operandos son positivos y el resultado es negativo.
	\item Los operandos son negativos y el resultado es positivo.
	\end{itemize}
\end{itemize}

\subsubsection{SUB}
Representa la instrucción SUB de la arquitectura Q. Es subclase de FlagInstruction. Sobreescribe los siguientes métodos:
\begin{itemize}
\item perform\_operation: recibe los operandos de origen y destino y devuelve la resta del origen al destino.
\item calculate\_overflow: recibe los operandos de origen y destino y el resultado de la operación. Devuelve verdadero si se cumple alguna de las siguientes condiciones: 
	\begin{itemize}
	\item El destino es negativo, el origen es positivo y el resultado positivo.
	\item El destino es positivo, el origen es negativo y el resultado negativo.
	\end{itemize}
\end{itemize}

\subsubsection{CMP}
Representa la instrucción CMP de la arquitectura Q. Es subclase de SUB. Sobreescribe el siguiente método:
\begin{itemize}
\item execute\_with\_state: recibe el estado y realiza la resta del origen al destino, calculando los flags. No guarda el resultado en el destino.
\end{itemize}

\subsubsection{MOV}
Representa la instrucción MOV de la arquitectura Q. Es subclase de Instruction. Sobreescribe los siguientes métodos:
\begin{itemize}
\item perform\_operation: recibe los operandos destino y origen y devuelve el origen.
\item read\_values: recibe el estado y devuelve el resultado de leer el operando origen. No lee el destino.
\end{itemize}

\subsubsection{MUL}
Representa la instrucción MUL de la arquitectura Q. Es subclase de FlagInstruction. Sobreescribe el siguiente método:
\begin{itemize}
\item perform\_operation: recibe los operandos de origen y destino y devuelve el producto de los mismos.
\end{itemize}

\subsubsection{DIV}
Representa la instrucción DIV de la arquitectura Q. Es subclase de FlagInstruction. Sobreescribe el siguiente método:
\begin{itemize}
\item perform\_operation: recibe los operandos de origen y destino y devuelve la división entera del destino por el origen. Si el origen es 0 lanza una excepción de tipo DivideByZeroError.
\end{itemize}

\subsubsection{AND}
Representa la instrucción AND de la arquitectura Q. Es subclase de FlagInstruction. Sobreescribe el siguiente método:
\begin{itemize}
\item perform\_operation: recibe los operandos de origen y destino y devuelve el producto lógico de los mismos.
\end{itemize}

\subsubsection{OR}
Representa la instrucción OR de la arquitectura Q. Es subclase de FlagInstruction. Sobreescribe el siguiente método:
\begin{itemize}
\item perform\_operation: recibe los operandos de origen y destino y devuelve la suma lógica de los mismos.
\end{itemize}



\subsubsection{CALL}
Representa la instrucción CALL de la arquitectura Q. Es subclase de SourceOnlyInstruction. Sobreescribe el siguiente método:
\begin{itemize}
\item execute\_with\_state: recibe el estado y escribe en pila el valor actual del pc, asigna el valor del origen al PC y disminuye el SP en 1.
\end{itemize}

\subsubsection{JMP}
Representa la instrucción JMP de la arquitectura Q. Es subclase de SourceOnlyInstruction. Sobreescribe el siguiente método:
\begin{itemize}
\item execute\_with\_state: recibe el estado y asigna el valor del origen al PC.
\end{itemize}




\subsubsection{JE}
Representa la instrucción JE de la arquitectura Q. Es subclase de ConditionalJump. Sobreescribe el siguiente método:
\begin{itemize}
\item condition\_matches: recibe el estado y devuelve el valor del flag Z.
\end{itemize}

\subsubsection{JNE}
Representa la instrucción JNE de la arquitectura Q. Es subclase de ConditionalJump. Sobreescribe el siguiente método:
\begin{itemize}
\item condition\_matches: recibe el estado y devuelve el valor negado del flag Z.
\end{itemize}

\subsubsection{JLE}
\label{subsubsec:jle}

Representa la instrucción JLE de la arquitectura Q. Es subclase de ConditionalJump. Sobreescribe el siguiente método:
\begin{itemize}
\item condition\_matches: recibe el estado y devuelve true si el flag Z es true o se da que o el flag N es verdadero o lo es el flag V (disyunción exclusiva), es decir, si se cumple la fórmula: \[Z + (N \oplus V) \]
\end{itemize}

\subsubsection{JG}
Representa la instrucción JG de la arquitectura Q. Es subclase de ConditionalJump. Sobreescribe el siguiente método:
\begin{itemize}
\item condition\_matches: recibe el estado y devuelve true no se cumple la condición de JLE \referencia{subsubsec:jle}, es decir, si se cumple la fórmula: \[\neg(Z + (N \oplus V)) \]
\end{itemize}

\subsubsection{JL}
\label{subsubsec:jl}

Representa la instrucción JL de la arquitectura Q. Es subclase de ConditionalJump. Sobreescribe el siguiente método:
\begin{itemize}
\item condition\_matches: recibe el estado y devuelve true si se cumple que o el flag N es verdadero o lo es el flag V (disyunción exclusiva), es decir, si se cumple la fórmula: \[N \oplus V \]
\end{itemize}

\subsubsection{JGE}
Representa la instrucción JGE de la arquitectura Q. Es subclase de ConditionalJump. Sobreescribe el siguiente método:
\begin{itemize}
\item condition\_matches: recibe el estado y devuelve true si no se cumple la condición de JL \referencia{subsubsec:jl}, es decir, si se cumple la fórmula: \[\neg(N \oplus V) \]
\end{itemize}

\subsubsection{JLEU}
Representa la instrucción JLEU de la arquitectura Q. Es subclase de ConditionalJump. Sobreescribe el siguiente método:
\begin{itemize}
\item condition\_matches: recibe el estado y devuelve true si se cumple que el flag C es verdadero o lo es el flag Z (disyunción inclusiva), es decir, si se cumple la fórmula: \[C + Z \]
\end{itemize}

\subsubsection{JGU}
Representa la instrucción JGU de la arquitectura Q. Es subclase de ConditionalJump. Sobreescribe el siguiente método:
\begin{itemize}
\item condition\_matches: recibe el estado y devuelve true si se cumple que ni el flag C es verdadero ni lo es el flag Z, es decir, si se cumple la fórmula: \[\neg(C + Z) \]
\end{itemize}

\subsubsection{JCS}
Representa la instrucción JCS de la arquitectura Q. Es subclase de ConditionalJump. Sobreescribe el siguiente método:
\begin{itemize}
\item condition\_matches: recibe el estado y devuelve el valor del flag C.
\end{itemize}

\subsubsection{JNEG}
Representa la instrucción JNEG de la arquitectura Q. Es subclase de ConditionalJump. Sobreescribe el siguiente método:
\begin{itemize}
\item condition\_matches: recibe el estado y devuelve el valor del flag N.
\end{itemize}

\subsubsection{JVS}
Representa la instrucción JVS de la arquitectura Q. Es subclase de ConditionalJump. Sobreescribe el siguiente método:
\begin{itemize}
\item condition\_matches: recibe el estado y devuelve el valor del flag V.
\end{itemize}






\subsubsection{NOT}
Representa la instrucción NOT de la arquitectura Q. Es subclase de FlagInstruction. Sobreescribe los siguientes métodos:
\begin{itemize}
\item perform\_operation: recibe los operandos destino y origen y devuelve el destino negado logicamente.
\item read\_values: recibe el estado y devuelve el resultado de leer el operando destino. No lee el origen.
\item disassemble: devuelve una instancia de la clase decodificando sólo el operando destino.
\end{itemize}

\subsubsection{RET}
Representa la instrucción RET de la arquitectura Q. Es subclase de Instruction. Sobreescribe los siguientes métodos:
\begin{itemize}
\item execute\_with\_state: recibe el estado e incrementa el valor del SP en 1 y lee de memoria la dirección a la que apunta el SP.
\item disassemble: devuelve una instancia de la clase sin utilizar ningún parámetro.
\end{itemize}
