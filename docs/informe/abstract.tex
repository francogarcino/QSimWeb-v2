\begin{abstract}
La arquitectura Q fue creada en la Universidad Nacional de Quilmes con el propósito de facilitar la enseñanza y el aprendizaje de lenguajes de bajo nivel, 
como assembler, en la materia Organización de Computadoras. Esta es una arquitectura conceptual, es decir, no existe una máquina física que funcione sobre dicha arquitectura.
Al ser Organización de Computadoras una materia inicial, los estudiantes encuentran una dificultad en el aprendizaje, marcado principalmente por la poca abstracción con la que cuentan estos lenguajes, sumando en nuestro caso, la imposibilidad de ejecutar sus programas. En este sentido, QSim, un trabajo de inserción 
profesional presentado en la misma universidad, aborda el problema brindando un entorno de ejecución para el lenguaje Q. Dicho proyecto fue encarado con 
una aplicación de escritorio escrita en Java que permitía ver los componentes de la arquitectura Q en funcionamiento. Algunos de los problemas que se 
encontraron a esta implementación fueron la baja performance de la herramienta y la dificultad que encuentran los estudiantes para configurar el entorno
 que requiere la aplicación para funcionar. 
 
Es por eso que en este trabajo se busca brindar una solución a este problema: un simulador de la arquitectura Q que permita visualizar de manera didáctica el proceso de 
ejecución de un lenguaje de bajo nivel pero realizada en un entorno web, lo cual evita las complejidades accidentales encontradas y nombradas 
anteriormente y agrega portabilidad para diversos dispositivos como celulares o tablets. Además, en contexto de la Pandemia por COVID-19, donde la 
educación en la Universidad tuvo un viraje a la educación virtual, es fundamental contar con una herramienta que pueda ser accedida mediante Internet.
El trabajo brinda una librería escrita en javascript que implementa el lenguaje Q y una interfaz visual que utiliza esa librería para mostrar de manera
 didáctica el proceso de ejecución. Esta separación permite realizar cambios en la interfaz de usuario sin afectar a la lógica del lenguaje.
\end{abstract}

\renewcommand{\abstractname}{Abstract}
\begin{abstract}
The Q architecture was created at the Universidad Nacional de Quilmes with the purpose of helping to teach and learn low-level languages, such as Assembly, in the context of the course of study Computer Organization.
It's a conceptual architecture, meaning that there is not a physical computer that works with it. 
Students often face a learning difficulty caused mainly by the lack of abstraction that these languages have, adding in our case the fact that Computer Organization is an initial subject and the impossibility of executing their programs.
Given this, QSim, a TIP presented at the same university, addresses the problem by providing an execution environment for the Q language.
The project was developed as a Java desktop application, which made it resource hungry and hard to set up.

Therefore this project tries to provide a solution to this problem: 
a tool that allows to didactically visualize the process of executing a low-level language while running on a web environment, 
which avoids the accidental complexities found in QSim and that also adds portability for various devices such as cell phones or tablets.
Moreover, in the context of a pandemic due to COVID-19, and universities embracing virtual learning, an online tool becomes even more significant.
This project provides a library written in javascript that implements the Q language and a visual interface that uses said library to display the execution process in a more didactic way. 
This separation allows changes to be made to the user interface without altering the language’s logic.
\end{abstract}
